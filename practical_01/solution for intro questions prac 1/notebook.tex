
% Default to the notebook output style

    


% Inherit from the specified cell style.




    
\documentclass[11pt]{article}

    
    
    \usepackage[T1]{fontenc}
    % Nicer default font (+ math font) than Computer Modern for most use cases
    \usepackage{mathpazo}

    % Basic figure setup, for now with no caption control since it's done
    % automatically by Pandoc (which extracts ![](path) syntax from Markdown).
    \usepackage{graphicx}
    % We will generate all images so they have a width \maxwidth. This means
    % that they will get their normal width if they fit onto the page, but
    % are scaled down if they would overflow the margins.
    \makeatletter
    \def\maxwidth{\ifdim\Gin@nat@width>\linewidth\linewidth
    \else\Gin@nat@width\fi}
    \makeatother
    \let\Oldincludegraphics\includegraphics
    % Set max figure width to be 80% of text width, for now hardcoded.
    \renewcommand{\includegraphics}[1]{\Oldincludegraphics[width=.8\maxwidth]{#1}}
    % Ensure that by default, figures have no caption (until we provide a
    % proper Figure object with a Caption API and a way to capture that
    % in the conversion process - todo).
    \usepackage{caption}
    \DeclareCaptionLabelFormat{nolabel}{}
    \captionsetup{labelformat=nolabel}

    \usepackage{adjustbox} % Used to constrain images to a maximum size 
    \usepackage{xcolor} % Allow colors to be defined
    \usepackage{enumerate} % Needed for markdown enumerations to work
    \usepackage{geometry} % Used to adjust the document margins
    \usepackage{amsmath} % Equations
    \usepackage{amssymb} % Equations
    \usepackage{textcomp} % defines textquotesingle
    % Hack from http://tex.stackexchange.com/a/47451/13684:
    \AtBeginDocument{%
        \def\PYZsq{\textquotesingle}% Upright quotes in Pygmentized code
    }
    \usepackage{upquote} % Upright quotes for verbatim code
    \usepackage{eurosym} % defines \euro
    \usepackage[mathletters]{ucs} % Extended unicode (utf-8) support
    \usepackage[utf8x]{inputenc} % Allow utf-8 characters in the tex document
    \usepackage{fancyvrb} % verbatim replacement that allows latex
    \usepackage{grffile} % extends the file name processing of package graphics 
                         % to support a larger range 
    % The hyperref package gives us a pdf with properly built
    % internal navigation ('pdf bookmarks' for the table of contents,
    % internal cross-reference links, web links for URLs, etc.)
    \usepackage{hyperref}
    \usepackage{longtable} % longtable support required by pandoc >1.10
    \usepackage{booktabs}  % table support for pandoc > 1.12.2
    \usepackage[inline]{enumitem} % IRkernel/repr support (it uses the enumerate* environment)
    \usepackage[normalem]{ulem} % ulem is needed to support strikethroughs (\sout)
                                % normalem makes italics be italics, not underlines
    

    
    
    % Colors for the hyperref package
    \definecolor{urlcolor}{rgb}{0,.145,.698}
    \definecolor{linkcolor}{rgb}{.71,0.21,0.01}
    \definecolor{citecolor}{rgb}{.12,.54,.11}

    % ANSI colors
    \definecolor{ansi-black}{HTML}{3E424D}
    \definecolor{ansi-black-intense}{HTML}{282C36}
    \definecolor{ansi-red}{HTML}{E75C58}
    \definecolor{ansi-red-intense}{HTML}{B22B31}
    \definecolor{ansi-green}{HTML}{00A250}
    \definecolor{ansi-green-intense}{HTML}{007427}
    \definecolor{ansi-yellow}{HTML}{DDB62B}
    \definecolor{ansi-yellow-intense}{HTML}{B27D12}
    \definecolor{ansi-blue}{HTML}{208FFB}
    \definecolor{ansi-blue-intense}{HTML}{0065CA}
    \definecolor{ansi-magenta}{HTML}{D160C4}
    \definecolor{ansi-magenta-intense}{HTML}{A03196}
    \definecolor{ansi-cyan}{HTML}{60C6C8}
    \definecolor{ansi-cyan-intense}{HTML}{258F8F}
    \definecolor{ansi-white}{HTML}{C5C1B4}
    \definecolor{ansi-white-intense}{HTML}{A1A6B2}

    % commands and environments needed by pandoc snippets
    % extracted from the output of `pandoc -s`
    \providecommand{\tightlist}{%
      \setlength{\itemsep}{0pt}\setlength{\parskip}{0pt}}
    \DefineVerbatimEnvironment{Highlighting}{Verbatim}{commandchars=\\\{\}}
    % Add ',fontsize=\small' for more characters per line
    \newenvironment{Shaded}{}{}
    \newcommand{\KeywordTok}[1]{\textcolor[rgb]{0.00,0.44,0.13}{\textbf{{#1}}}}
    \newcommand{\DataTypeTok}[1]{\textcolor[rgb]{0.56,0.13,0.00}{{#1}}}
    \newcommand{\DecValTok}[1]{\textcolor[rgb]{0.25,0.63,0.44}{{#1}}}
    \newcommand{\BaseNTok}[1]{\textcolor[rgb]{0.25,0.63,0.44}{{#1}}}
    \newcommand{\FloatTok}[1]{\textcolor[rgb]{0.25,0.63,0.44}{{#1}}}
    \newcommand{\CharTok}[1]{\textcolor[rgb]{0.25,0.44,0.63}{{#1}}}
    \newcommand{\StringTok}[1]{\textcolor[rgb]{0.25,0.44,0.63}{{#1}}}
    \newcommand{\CommentTok}[1]{\textcolor[rgb]{0.38,0.63,0.69}{\textit{{#1}}}}
    \newcommand{\OtherTok}[1]{\textcolor[rgb]{0.00,0.44,0.13}{{#1}}}
    \newcommand{\AlertTok}[1]{\textcolor[rgb]{1.00,0.00,0.00}{\textbf{{#1}}}}
    \newcommand{\FunctionTok}[1]{\textcolor[rgb]{0.02,0.16,0.49}{{#1}}}
    \newcommand{\RegionMarkerTok}[1]{{#1}}
    \newcommand{\ErrorTok}[1]{\textcolor[rgb]{1.00,0.00,0.00}{\textbf{{#1}}}}
    \newcommand{\NormalTok}[1]{{#1}}
    
    % Additional commands for more recent versions of Pandoc
    \newcommand{\ConstantTok}[1]{\textcolor[rgb]{0.53,0.00,0.00}{{#1}}}
    \newcommand{\SpecialCharTok}[1]{\textcolor[rgb]{0.25,0.44,0.63}{{#1}}}
    \newcommand{\VerbatimStringTok}[1]{\textcolor[rgb]{0.25,0.44,0.63}{{#1}}}
    \newcommand{\SpecialStringTok}[1]{\textcolor[rgb]{0.73,0.40,0.53}{{#1}}}
    \newcommand{\ImportTok}[1]{{#1}}
    \newcommand{\DocumentationTok}[1]{\textcolor[rgb]{0.73,0.13,0.13}{\textit{{#1}}}}
    \newcommand{\AnnotationTok}[1]{\textcolor[rgb]{0.38,0.63,0.69}{\textbf{\textit{{#1}}}}}
    \newcommand{\CommentVarTok}[1]{\textcolor[rgb]{0.38,0.63,0.69}{\textbf{\textit{{#1}}}}}
    \newcommand{\VariableTok}[1]{\textcolor[rgb]{0.10,0.09,0.49}{{#1}}}
    \newcommand{\ControlFlowTok}[1]{\textcolor[rgb]{0.00,0.44,0.13}{\textbf{{#1}}}}
    \newcommand{\OperatorTok}[1]{\textcolor[rgb]{0.40,0.40,0.40}{{#1}}}
    \newcommand{\BuiltInTok}[1]{{#1}}
    \newcommand{\ExtensionTok}[1]{{#1}}
    \newcommand{\PreprocessorTok}[1]{\textcolor[rgb]{0.74,0.48,0.00}{{#1}}}
    \newcommand{\AttributeTok}[1]{\textcolor[rgb]{0.49,0.56,0.16}{{#1}}}
    \newcommand{\InformationTok}[1]{\textcolor[rgb]{0.38,0.63,0.69}{\textbf{\textit{{#1}}}}}
    \newcommand{\WarningTok}[1]{\textcolor[rgb]{0.38,0.63,0.69}{\textbf{\textit{{#1}}}}}
    
    
    % Define a nice break command that doesn't care if a line doesn't already
    % exist.
    \def\br{\hspace*{\fill} \\* }
    % Math Jax compatability definitions
    \def\gt{>}
    \def\lt{<}
    % Document parameters
    \title{prac1\_solution}
    
    
    

    % Pygments definitions
    
\makeatletter
\def\PY@reset{\let\PY@it=\relax \let\PY@bf=\relax%
    \let\PY@ul=\relax \let\PY@tc=\relax%
    \let\PY@bc=\relax \let\PY@ff=\relax}
\def\PY@tok#1{\csname PY@tok@#1\endcsname}
\def\PY@toks#1+{\ifx\relax#1\empty\else%
    \PY@tok{#1}\expandafter\PY@toks\fi}
\def\PY@do#1{\PY@bc{\PY@tc{\PY@ul{%
    \PY@it{\PY@bf{\PY@ff{#1}}}}}}}
\def\PY#1#2{\PY@reset\PY@toks#1+\relax+\PY@do{#2}}

\expandafter\def\csname PY@tok@w\endcsname{\def\PY@tc##1{\textcolor[rgb]{0.73,0.73,0.73}{##1}}}
\expandafter\def\csname PY@tok@c\endcsname{\let\PY@it=\textit\def\PY@tc##1{\textcolor[rgb]{0.25,0.50,0.50}{##1}}}
\expandafter\def\csname PY@tok@cp\endcsname{\def\PY@tc##1{\textcolor[rgb]{0.74,0.48,0.00}{##1}}}
\expandafter\def\csname PY@tok@k\endcsname{\let\PY@bf=\textbf\def\PY@tc##1{\textcolor[rgb]{0.00,0.50,0.00}{##1}}}
\expandafter\def\csname PY@tok@kp\endcsname{\def\PY@tc##1{\textcolor[rgb]{0.00,0.50,0.00}{##1}}}
\expandafter\def\csname PY@tok@kt\endcsname{\def\PY@tc##1{\textcolor[rgb]{0.69,0.00,0.25}{##1}}}
\expandafter\def\csname PY@tok@o\endcsname{\def\PY@tc##1{\textcolor[rgb]{0.40,0.40,0.40}{##1}}}
\expandafter\def\csname PY@tok@ow\endcsname{\let\PY@bf=\textbf\def\PY@tc##1{\textcolor[rgb]{0.67,0.13,1.00}{##1}}}
\expandafter\def\csname PY@tok@nb\endcsname{\def\PY@tc##1{\textcolor[rgb]{0.00,0.50,0.00}{##1}}}
\expandafter\def\csname PY@tok@nf\endcsname{\def\PY@tc##1{\textcolor[rgb]{0.00,0.00,1.00}{##1}}}
\expandafter\def\csname PY@tok@nc\endcsname{\let\PY@bf=\textbf\def\PY@tc##1{\textcolor[rgb]{0.00,0.00,1.00}{##1}}}
\expandafter\def\csname PY@tok@nn\endcsname{\let\PY@bf=\textbf\def\PY@tc##1{\textcolor[rgb]{0.00,0.00,1.00}{##1}}}
\expandafter\def\csname PY@tok@ne\endcsname{\let\PY@bf=\textbf\def\PY@tc##1{\textcolor[rgb]{0.82,0.25,0.23}{##1}}}
\expandafter\def\csname PY@tok@nv\endcsname{\def\PY@tc##1{\textcolor[rgb]{0.10,0.09,0.49}{##1}}}
\expandafter\def\csname PY@tok@no\endcsname{\def\PY@tc##1{\textcolor[rgb]{0.53,0.00,0.00}{##1}}}
\expandafter\def\csname PY@tok@nl\endcsname{\def\PY@tc##1{\textcolor[rgb]{0.63,0.63,0.00}{##1}}}
\expandafter\def\csname PY@tok@ni\endcsname{\let\PY@bf=\textbf\def\PY@tc##1{\textcolor[rgb]{0.60,0.60,0.60}{##1}}}
\expandafter\def\csname PY@tok@na\endcsname{\def\PY@tc##1{\textcolor[rgb]{0.49,0.56,0.16}{##1}}}
\expandafter\def\csname PY@tok@nt\endcsname{\let\PY@bf=\textbf\def\PY@tc##1{\textcolor[rgb]{0.00,0.50,0.00}{##1}}}
\expandafter\def\csname PY@tok@nd\endcsname{\def\PY@tc##1{\textcolor[rgb]{0.67,0.13,1.00}{##1}}}
\expandafter\def\csname PY@tok@s\endcsname{\def\PY@tc##1{\textcolor[rgb]{0.73,0.13,0.13}{##1}}}
\expandafter\def\csname PY@tok@sd\endcsname{\let\PY@it=\textit\def\PY@tc##1{\textcolor[rgb]{0.73,0.13,0.13}{##1}}}
\expandafter\def\csname PY@tok@si\endcsname{\let\PY@bf=\textbf\def\PY@tc##1{\textcolor[rgb]{0.73,0.40,0.53}{##1}}}
\expandafter\def\csname PY@tok@se\endcsname{\let\PY@bf=\textbf\def\PY@tc##1{\textcolor[rgb]{0.73,0.40,0.13}{##1}}}
\expandafter\def\csname PY@tok@sr\endcsname{\def\PY@tc##1{\textcolor[rgb]{0.73,0.40,0.53}{##1}}}
\expandafter\def\csname PY@tok@ss\endcsname{\def\PY@tc##1{\textcolor[rgb]{0.10,0.09,0.49}{##1}}}
\expandafter\def\csname PY@tok@sx\endcsname{\def\PY@tc##1{\textcolor[rgb]{0.00,0.50,0.00}{##1}}}
\expandafter\def\csname PY@tok@m\endcsname{\def\PY@tc##1{\textcolor[rgb]{0.40,0.40,0.40}{##1}}}
\expandafter\def\csname PY@tok@gh\endcsname{\let\PY@bf=\textbf\def\PY@tc##1{\textcolor[rgb]{0.00,0.00,0.50}{##1}}}
\expandafter\def\csname PY@tok@gu\endcsname{\let\PY@bf=\textbf\def\PY@tc##1{\textcolor[rgb]{0.50,0.00,0.50}{##1}}}
\expandafter\def\csname PY@tok@gd\endcsname{\def\PY@tc##1{\textcolor[rgb]{0.63,0.00,0.00}{##1}}}
\expandafter\def\csname PY@tok@gi\endcsname{\def\PY@tc##1{\textcolor[rgb]{0.00,0.63,0.00}{##1}}}
\expandafter\def\csname PY@tok@gr\endcsname{\def\PY@tc##1{\textcolor[rgb]{1.00,0.00,0.00}{##1}}}
\expandafter\def\csname PY@tok@ge\endcsname{\let\PY@it=\textit}
\expandafter\def\csname PY@tok@gs\endcsname{\let\PY@bf=\textbf}
\expandafter\def\csname PY@tok@gp\endcsname{\let\PY@bf=\textbf\def\PY@tc##1{\textcolor[rgb]{0.00,0.00,0.50}{##1}}}
\expandafter\def\csname PY@tok@go\endcsname{\def\PY@tc##1{\textcolor[rgb]{0.53,0.53,0.53}{##1}}}
\expandafter\def\csname PY@tok@gt\endcsname{\def\PY@tc##1{\textcolor[rgb]{0.00,0.27,0.87}{##1}}}
\expandafter\def\csname PY@tok@err\endcsname{\def\PY@bc##1{\setlength{\fboxsep}{0pt}\fcolorbox[rgb]{1.00,0.00,0.00}{1,1,1}{\strut ##1}}}
\expandafter\def\csname PY@tok@kc\endcsname{\let\PY@bf=\textbf\def\PY@tc##1{\textcolor[rgb]{0.00,0.50,0.00}{##1}}}
\expandafter\def\csname PY@tok@kd\endcsname{\let\PY@bf=\textbf\def\PY@tc##1{\textcolor[rgb]{0.00,0.50,0.00}{##1}}}
\expandafter\def\csname PY@tok@kn\endcsname{\let\PY@bf=\textbf\def\PY@tc##1{\textcolor[rgb]{0.00,0.50,0.00}{##1}}}
\expandafter\def\csname PY@tok@kr\endcsname{\let\PY@bf=\textbf\def\PY@tc##1{\textcolor[rgb]{0.00,0.50,0.00}{##1}}}
\expandafter\def\csname PY@tok@bp\endcsname{\def\PY@tc##1{\textcolor[rgb]{0.00,0.50,0.00}{##1}}}
\expandafter\def\csname PY@tok@fm\endcsname{\def\PY@tc##1{\textcolor[rgb]{0.00,0.00,1.00}{##1}}}
\expandafter\def\csname PY@tok@vc\endcsname{\def\PY@tc##1{\textcolor[rgb]{0.10,0.09,0.49}{##1}}}
\expandafter\def\csname PY@tok@vg\endcsname{\def\PY@tc##1{\textcolor[rgb]{0.10,0.09,0.49}{##1}}}
\expandafter\def\csname PY@tok@vi\endcsname{\def\PY@tc##1{\textcolor[rgb]{0.10,0.09,0.49}{##1}}}
\expandafter\def\csname PY@tok@vm\endcsname{\def\PY@tc##1{\textcolor[rgb]{0.10,0.09,0.49}{##1}}}
\expandafter\def\csname PY@tok@sa\endcsname{\def\PY@tc##1{\textcolor[rgb]{0.73,0.13,0.13}{##1}}}
\expandafter\def\csname PY@tok@sb\endcsname{\def\PY@tc##1{\textcolor[rgb]{0.73,0.13,0.13}{##1}}}
\expandafter\def\csname PY@tok@sc\endcsname{\def\PY@tc##1{\textcolor[rgb]{0.73,0.13,0.13}{##1}}}
\expandafter\def\csname PY@tok@dl\endcsname{\def\PY@tc##1{\textcolor[rgb]{0.73,0.13,0.13}{##1}}}
\expandafter\def\csname PY@tok@s2\endcsname{\def\PY@tc##1{\textcolor[rgb]{0.73,0.13,0.13}{##1}}}
\expandafter\def\csname PY@tok@sh\endcsname{\def\PY@tc##1{\textcolor[rgb]{0.73,0.13,0.13}{##1}}}
\expandafter\def\csname PY@tok@s1\endcsname{\def\PY@tc##1{\textcolor[rgb]{0.73,0.13,0.13}{##1}}}
\expandafter\def\csname PY@tok@mb\endcsname{\def\PY@tc##1{\textcolor[rgb]{0.40,0.40,0.40}{##1}}}
\expandafter\def\csname PY@tok@mf\endcsname{\def\PY@tc##1{\textcolor[rgb]{0.40,0.40,0.40}{##1}}}
\expandafter\def\csname PY@tok@mh\endcsname{\def\PY@tc##1{\textcolor[rgb]{0.40,0.40,0.40}{##1}}}
\expandafter\def\csname PY@tok@mi\endcsname{\def\PY@tc##1{\textcolor[rgb]{0.40,0.40,0.40}{##1}}}
\expandafter\def\csname PY@tok@il\endcsname{\def\PY@tc##1{\textcolor[rgb]{0.40,0.40,0.40}{##1}}}
\expandafter\def\csname PY@tok@mo\endcsname{\def\PY@tc##1{\textcolor[rgb]{0.40,0.40,0.40}{##1}}}
\expandafter\def\csname PY@tok@ch\endcsname{\let\PY@it=\textit\def\PY@tc##1{\textcolor[rgb]{0.25,0.50,0.50}{##1}}}
\expandafter\def\csname PY@tok@cm\endcsname{\let\PY@it=\textit\def\PY@tc##1{\textcolor[rgb]{0.25,0.50,0.50}{##1}}}
\expandafter\def\csname PY@tok@cpf\endcsname{\let\PY@it=\textit\def\PY@tc##1{\textcolor[rgb]{0.25,0.50,0.50}{##1}}}
\expandafter\def\csname PY@tok@c1\endcsname{\let\PY@it=\textit\def\PY@tc##1{\textcolor[rgb]{0.25,0.50,0.50}{##1}}}
\expandafter\def\csname PY@tok@cs\endcsname{\let\PY@it=\textit\def\PY@tc##1{\textcolor[rgb]{0.25,0.50,0.50}{##1}}}

\def\PYZbs{\char`\\}
\def\PYZus{\char`\_}
\def\PYZob{\char`\{}
\def\PYZcb{\char`\}}
\def\PYZca{\char`\^}
\def\PYZam{\char`\&}
\def\PYZlt{\char`\<}
\def\PYZgt{\char`\>}
\def\PYZsh{\char`\#}
\def\PYZpc{\char`\%}
\def\PYZdl{\char`\$}
\def\PYZhy{\char`\-}
\def\PYZsq{\char`\'}
\def\PYZdq{\char`\"}
\def\PYZti{\char`\~}
% for compatibility with earlier versions
\def\PYZat{@}
\def\PYZlb{[}
\def\PYZrb{]}
\makeatother


    % Exact colors from NB
    \definecolor{incolor}{rgb}{0.0, 0.0, 0.5}
    \definecolor{outcolor}{rgb}{0.545, 0.0, 0.0}



    
    % Prevent overflowing lines due to hard-to-break entities
    \sloppy 
    % Setup hyperref package
    \hypersetup{
      breaklinks=true,  % so long urls are correctly broken across lines
      colorlinks=true,
      urlcolor=urlcolor,
      linkcolor=linkcolor,
      citecolor=citecolor,
      }
    % Slightly bigger margins than the latex defaults
    
    \geometry{verbose,tmargin=1in,bmargin=1in,lmargin=1in,rmargin=1in}
    
    

    \begin{document}
    
    
    \maketitle
    
    

    
    \section{Introduction to Data Mining}\label{introduction-to-data-mining}

    This is the first tutorial - it provides a quick introduction to Python
and IPython.

    \subsubsection{Python}\label{python}

    Python is a programming language that has been growing in popularity in
recent years. There are many reasons for this, but it mostly comes down
to Python being easy to learn and use as well as the fact that Python
has a very active community that develops amazing extensions to Python!

Python has become one of the most frequently used languages in the world
of data science due to the ability to almost instantly apply it to a
large number of data science problems. When asking companies in
different industries and of various sizes what lanuage they would like
their data scientists to know when coming in, they almost all agree that
Python is the best choice.

Most of you have had experience in Java. Both share many characteristics
given they are both programming languages and many structures and
concepts will be familiar. However, there are differences. You may find
you prefer python and it may allow you to be more productive, not just
because of the provision of many libraries which implement Data Mining
algorithms.

Some key differences between Python and Java are: - Python is
dynamically typed while Java is statically typed, this has various
implications including you never need to declare variables in Python -
Python code is more concise (briefer) - Python is more compact

For example, the following program is in Java: public class HelloWorld
\{ public static void main (String{[}{]} args) \{
System.out.println("Hellold!"); \} \}

In Python the equivalent program is print("Hello, world!") \# Python
version 3

    \subsubsection{Python? IPython? IPython notebooks? What is all of
this?}\label{python-ipython-ipython-notebooks-what-is-all-of-this}

    What is all of this? - Python? - IPython? - IPython notebooks?

For you, these terms may be confusing? Don't worry its ok...

Python is a lanuage. A computer programming language. It is very popular
in lots of industries. It has been around for over 30 years. Development
of Python started in December 1989 by Guido van Rossum at Centrum
Wiskunde \& Informatica (CWI) in the Netherlands.

IPython (with an I) is an extension to Python. It was created by
scientists for scientists. It is typically used in computer science,
machine learning, and physics research. It simply adds some new features
to Python. Writing code in either of these is generally done in a long
text document with a ton of code that can be pretty duanting to a
someone new to the world of programming.

IPython Notebooks (with an I, there is no such thing as a Python
notebook) is what we are looking at right now. It is a web browser based
way of writing Python code. One of the benefits is that it allows you to
write in plain text to create, what should feel like, a notebook. The
closest analog here would be to relate an IPython notebook to a typical
lab notebook kept by "traditional" researchers. Anyone coming from
chemistry or biology will probably understand what I mean. We will be
using IPython notebooks for the rest of the semester. A majority of your
class notes will be presented in this format. This will allow us to both
have a place for discussion and instruction, but with the added benefit
of allowing us to play with data live!

The rest of this document will be broken into two main pieces: an
introduction to IPython notebooks (how to use them) and then an
introduction to the world of Python programming.

    \subsubsection{IPython Notebooks}\label{ipython-notebooks}

    IPython notebooks are made up of cells. There are two basics types of
entries in an IPython notebook: text cells, and code cells.

You can edit a cell by double clicking on it. You can get it back to the
display mode by pressing the "Run" button from the toolbar. Try it! To
switch between text and code cells, just click a cell and go to "Cell
\textgreater{} Cell Type" in the menu bar (or use the toolbar).

    \paragraph{Text Cells}\label{text-cells}

    Let's start a new cell and add a little bit more text.

You can do text formatting. For example, you can use asterisks or
underscores to emphasize things. Double asterisks are used to make
things bold.

If you know what LaTeX is, you can write directly in LaTeX by wrapping
it in dollar signs, \(x = \frac{-b \pm \sqrt{b^2 - 4ac}}{2a}\). Learning
\(\LaTeX\) is great for typesetting math formulas.

Creating a bulleted list is pretty easy: - One - Two - Three

It is also very easy to make a numbered list: 1. One 1. Two 1. Three

For a numbered list, start with 1. followed by a space, then it starts
numbering for you. Start each line with some number (any number) and a
period, then a space. Tab to indent to get subnumbering.

This covers almost all of the text formatting you will need to learn. If
you are ever stuck, just Google "Markdown syntax" since the language the
formatting is done in is called Markdown.

    \paragraph{Code Cells}\label{code-cells}

    Now, we will see a "code" cell. Here, we simply type any Python code and
then click "Run". When we run a cell, the code in it is executed and
remembered for as long as we keep this window open - this means if you
create a variable in a code cell you can refer to the variable by its
name in other subsequent cells. Code cells will always start with "In
{[} {]}:". Code cells are the default for any new cell.

    \begin{Verbatim}[commandchars=\\\{\}]
{\color{incolor}In [{\color{incolor}4}]:} \PY{n}{x} \PY{o}{=} \PY{l+m+mi}{5} \PY{o}{+} \PY{l+m+mi}{10}
\end{Verbatim}


    \begin{Verbatim}[commandchars=\\\{\}]
{\color{incolor}In [{\color{incolor}7}]:} \PY{n+nb}{print} \PY{p}{(}\PY{n}{x}\PY{p}{)}
\end{Verbatim}


    \begin{Verbatim}[commandchars=\\\{\}]
15

    \end{Verbatim}

    You can include more complex expressions or programs in a cell as well:

    \begin{Verbatim}[commandchars=\\\{\}]
{\color{incolor}In [{\color{incolor}9}]:} \PY{n}{x} \PY{o}{=} \PY{l+m+mi}{5} \PY{o}{+} \PY{l+m+mi}{1}
        \PY{n+nb}{print} \PY{p}{(}\PY{l+s+s2}{\PYZdq{}}\PY{l+s+s2}{The value of x is }\PY{l+s+s2}{\PYZdq{}} \PY{o}{+} \PY{n+nb}{str}\PY{p}{(}\PY{n}{x}\PY{p}{)} \PY{o}{+} \PY{l+s+s2}{\PYZdq{}}\PY{l+s+s2}{.}\PY{l+s+s2}{\PYZdq{}}\PY{p}{)}
        \PY{n+nb}{print} \PY{p}{(}\PY{l+s+s2}{\PYZdq{}}\PY{l+s+s2}{Output: Hi! This is a cell. Press the ▶| button above to run it}\PY{l+s+s2}{\PYZdq{}}\PY{p}{)}
\end{Verbatim}


    \begin{Verbatim}[commandchars=\\\{\}]
The value of x is 6.
Output: Hi! This is a cell. Press the ▶| button above to run it

    \end{Verbatim}

    Instead of clicking "Play" button, you can run a cell with Ctrl + Enter
or Shift + Enter. Experiment with both of those to see what the
difference is.

    \subsubsection{Python}\label{python}

    Let us now examine some of the key elements of Python

    \paragraph{Variables and data types}\label{variables-and-data-types}

    Variables are used to store data. This data can be of a variety of
types. Integer numbers, floating (decimal numbers), lists, strings, etc.
Let's take a look at some of these:

    \begin{Verbatim}[commandchars=\\\{\}]
{\color{incolor}In [{\color{incolor}8}]:} \PY{n}{some\PYZus{}integer} \PY{o}{=} \PY{l+m+mi}{5}
        \PY{n}{some\PYZus{}float} \PY{o}{=} \PY{l+m+mf}{7.1}
        \PY{n}{some\PYZus{}list} \PY{o}{=} \PY{p}{[}\PY{l+m+mi}{1}\PY{p}{,} \PY{l+m+mi}{2}\PY{p}{,} \PY{l+m+mi}{3}\PY{p}{,} \PY{l+m+mi}{4}\PY{p}{]}
        \PY{n}{some\PYZus{}string} \PY{o}{=} \PY{l+s+s2}{\PYZdq{}}\PY{l+s+s2}{Rob}\PY{l+s+s2}{\PYZdq{}}
\end{Verbatim}


    We can print out these variables.

    \begin{Verbatim}[commandchars=\\\{\}]
{\color{incolor}In [{\color{incolor}9}]:} \PY{n+nb}{print} \PY{p}{(}\PY{n}{some\PYZus{}integer}\PY{p}{)}
        \PY{n+nb}{print} \PY{p}{(}\PY{n}{some\PYZus{}float}\PY{p}{)}
        \PY{n+nb}{print} \PY{p}{(}\PY{n}{some\PYZus{}list}\PY{p}{)}
        \PY{n+nb}{print} \PY{p}{(}\PY{n}{some\PYZus{}string}\PY{p}{)}
\end{Verbatim}


    \begin{Verbatim}[commandchars=\\\{\}]
5
7.1
[1, 2, 3, 4]
Rob

    \end{Verbatim}

    What if I want to print some text and then some numbers? One easy way to
do this is to realize that printing will always want string data. If you
have data that is not a string (like an integer or float), you can
convert it to a string.

    print ("My integer was " + str(some\_integer) + ".")

    It is always a good practice to convert everything to a string when
printing it out such as by using the function str().

Now we look at some basic maths.

    \begin{Verbatim}[commandchars=\\\{\}]
{\color{incolor}In [{\color{incolor}12}]:} \PY{n}{some\PYZus{}integer} \PY{o}{+} \PY{n}{some\PYZus{}float}
\end{Verbatim}


\begin{Verbatim}[commandchars=\\\{\}]
{\color{outcolor}Out[{\color{outcolor}12}]:} 12.1
\end{Verbatim}
            
    We can store this as a new variable to use later,

    \begin{Verbatim}[commandchars=\\\{\}]
{\color{incolor}In [{\color{incolor}15}]:} \PY{n}{my\PYZus{}sum} \PY{o}{=} \PY{n}{some\PYZus{}integer} \PY{o}{+} \PY{n}{some\PYZus{}float}
\end{Verbatim}


    \begin{Verbatim}[commandchars=\\\{\}]
{\color{incolor}In [{\color{incolor}17}]:} \PY{n+nb}{print} \PY{p}{(}\PY{n}{my\PYZus{}sum}\PY{p}{)}
\end{Verbatim}


    \begin{Verbatim}[commandchars=\\\{\}]
12.1

    \end{Verbatim}

    What about that list we had? What does that mean? A list is exactly what
it sounds like. It's a way to keep a collection of things in order. We
can check to see how long our list is,

    \begin{Verbatim}[commandchars=\\\{\}]
{\color{incolor}In [{\color{incolor}19}]:} \PY{n+nb}{print}\PY{p}{(} \PY{n+nb}{len}\PY{p}{(}\PY{n}{some\PYZus{}list}\PY{p}{)} \PY{p}{)}
\end{Verbatim}


    \begin{Verbatim}[commandchars=\\\{\}]
4

    \end{Verbatim}

    This looks good. Our list contained the numbers 1 through 4. What if we
want a particular item from the list? How do we look at just the first
item? To do a lookup, we use square brackets. Notice that when we
created the list originally, we also used square brackets!

    \begin{Verbatim}[commandchars=\\\{\}]
{\color{incolor}In [{\color{incolor}20}]:} \PY{n+nb}{print}\PY{p}{(} \PY{n}{some\PYZus{}list}\PY{p}{[}\PY{l+m+mi}{1}\PY{p}{]} \PY{p}{)}
\end{Verbatim}


    \begin{Verbatim}[commandchars=\\\{\}]
2

    \end{Verbatim}

    That's the second item, not the first! In Python (and almost every other
language), counting start at zero! To get the first item we should look
in the 0th space,

    \begin{Verbatim}[commandchars=\\\{\}]
{\color{incolor}In [{\color{incolor}22}]:} \PY{n+nb}{print}\PY{p}{(} \PY{n}{some\PYZus{}list}\PY{p}{[}\PY{l+m+mi}{0}\PY{p}{]} \PY{p}{)}
\end{Verbatim}


    \begin{Verbatim}[commandchars=\\\{\}]
1

    \end{Verbatim}

    Adding things to the list is done using the append method,

    \begin{Verbatim}[commandchars=\\\{\}]
{\color{incolor}In [{\color{incolor}23}]:} \PY{n}{some\PYZus{}list}\PY{o}{.}\PY{n}{append}\PY{p}{(}\PY{l+m+mi}{5}\PY{p}{)}
\end{Verbatim}


    \begin{Verbatim}[commandchars=\\\{\}]
{\color{incolor}In [{\color{incolor}24}]:} \PY{n+nb}{print} \PY{p}{(}\PY{n}{some\PYZus{}list}\PY{p}{)}
\end{Verbatim}


    \begin{Verbatim}[commandchars=\\\{\}]
[1, 2, 3, 4, 5]

    \end{Verbatim}

    \begin{Verbatim}[commandchars=\\\{\}]
{\color{incolor}In [{\color{incolor} }]:} \PY{c+c1}{\PYZsh{} Play around here!}
        \PY{c+c1}{\PYZsh{} By the way, the pound (hash) symbol here is used to indicate a comment in code.}
\end{Verbatim}


    \paragraph{Functions}\label{functions}

    We have already used these twice! Functions allow us to do predefined
operations. Functions are usually some sensible English word ending in
open-and-close parentheses. One example is the s

    \begin{Verbatim}[commandchars=\\\{\}]
{\color{incolor}In [{\color{incolor}25}]:} \PY{n+nb}{str}\PY{p}{(}\PY{l+m+mf}{5.124}\PY{p}{)}
\end{Verbatim}


\begin{Verbatim}[commandchars=\\\{\}]
{\color{outcolor}Out[{\color{outcolor}25}]:} '5.124'
\end{Verbatim}
            
    We also used the append() function to add stuff to a list.

If we knew we had to do some operation many times, and wanted to save a
bit of time, we could define our own function. For example, consider
having to calculate the area of a circle.

    \begin{Verbatim}[commandchars=\\\{\}]
{\color{incolor}In [{\color{incolor}26}]:} \PY{k}{def} \PY{n+nf}{area\PYZus{}of\PYZus{}a\PYZus{}circle}\PY{p}{(}\PY{n}{radius}\PY{p}{)}\PY{p}{:}
             \PY{n}{area} \PY{o}{=} \PY{l+m+mf}{3.14} \PY{o}{*} \PY{n}{radius} \PY{o}{*} \PY{n}{radius}
             \PY{k}{return} \PY{n}{area}
\end{Verbatim}


    \begin{Verbatim}[commandchars=\\\{\}]
{\color{incolor}In [{\color{incolor}28}]:} \PY{n}{circle\PYZus{}area} \PY{o}{=} \PY{n}{area\PYZus{}of\PYZus{}a\PYZus{}circle}\PY{p}{(}\PY{l+m+mi}{5}\PY{p}{)}
         \PY{n+nb}{print}\PY{p}{(} \PY{n}{circle\PYZus{}area} \PY{p}{)}
\end{Verbatim}


    \begin{Verbatim}[commandchars=\\\{\}]
78.5

    \end{Verbatim}

    This function was helpfully named "area\_of\_a\_circle", it takes one
argument that we will call radius. It then uses this radius to get the
area and then returns it. Now, whenever I want to get the area of some
circle, I simply call area\_of\_a\_circle() and place the radius in the
middle of the parentheses.

Python has many functions, but we will be writing our own very often.

    \paragraph{Loops}\label{loops}

    We will be doing a lot of repetative things in Python. This doesn't mean
we need to do a ton of copy and pasting, though. We can use loops to
make this easy. For example, if we wanted to square each

    \begin{Verbatim}[commandchars=\\\{\}]
{\color{incolor}In [{\color{incolor}30}]:} \PY{k}{for} \PY{n}{number} \PY{o+ow}{in} \PY{p}{[}\PY{l+m+mi}{1}\PY{p}{,} \PY{l+m+mi}{2}\PY{p}{,} \PY{l+m+mi}{3}\PY{p}{,} \PY{l+m+mi}{4}\PY{p}{,} \PY{l+m+mi}{5}\PY{p}{]}\PY{p}{:}
             \PY{n+nb}{print} \PY{p}{(}\PY{n}{number} \PY{o}{*} \PY{n}{number}\PY{p}{)}
\end{Verbatim}


    \begin{Verbatim}[commandchars=\\\{\}]
1
4
9
16
25

    \end{Verbatim}

    The range function makes this even easier,

    \begin{Verbatim}[commandchars=\\\{\}]
{\color{incolor}In [{\color{incolor}31}]:} \PY{k}{for} \PY{n}{number} \PY{o+ow}{in} \PY{n+nb}{range}\PY{p}{(}\PY{l+m+mi}{5}\PY{p}{)}\PY{p}{:}
             \PY{n+nb}{print} \PY{p}{(}\PY{n}{number} \PY{o}{*} \PY{n}{number}\PY{p}{)}
\end{Verbatim}


    \begin{Verbatim}[commandchars=\\\{\}]
0
1
4
9
16

    \end{Verbatim}

    Not exactly the same... the range function will start from 0 and go to
the last number minus one. We can fix this by telling it to start at 1:

    \begin{Verbatim}[commandchars=\\\{\}]
{\color{incolor}In [{\color{incolor}33}]:} \PY{k}{for} \PY{n}{number} \PY{o+ow}{in} \PY{n+nb}{range}\PY{p}{(}\PY{l+m+mi}{1}\PY{p}{,} \PY{l+m+mi}{6}\PY{p}{)}\PY{p}{:}
             \PY{n+nb}{print} \PY{p}{(}\PY{n}{number} \PY{o}{*} \PY{n}{number}\PY{p}{)}
\end{Verbatim}


    \begin{Verbatim}[commandchars=\\\{\}]
1
4
9
16
25

    \end{Verbatim}

    We aren't limited to this, let's bring in another list:

    \begin{Verbatim}[commandchars=\\\{\}]
{\color{incolor}In [{\color{incolor}38}]:} \PY{n}{names} \PY{o}{=} \PY{p}{[}\PY{l+s+s2}{\PYZdq{}}\PY{l+s+s2}{Robert}\PY{l+s+s2}{\PYZdq{}}\PY{p}{,} \PY{l+s+s2}{\PYZdq{}}\PY{l+s+s2}{John}\PY{l+s+s2}{\PYZdq{}}\PY{p}{,} \PY{l+s+s2}{\PYZdq{}}\PY{l+s+s2}{Sarah}\PY{l+s+s2}{\PYZdq{}}\PY{p}{,} \PY{l+s+s2}{\PYZdq{}}\PY{l+s+s2}{Qian}\PY{l+s+s2}{\PYZdq{}}\PY{p}{,} \PY{l+s+s2}{\PYZdq{}}\PY{l+s+s2}{Ahmad}\PY{l+s+s2}{\PYZdq{}}\PY{p}{]}
         \PY{n}{ages} \PY{o}{=} \PY{p}{[}\PY{l+m+mi}{26}\PY{p}{,} \PY{l+m+mi}{31}\PY{p}{,} \PY{l+m+mi}{29}\PY{p}{,} \PY{l+m+mi}{24}\PY{p}{,} \PY{l+m+mi}{30}\PY{p}{]}
         
         \PY{k}{for} \PY{n}{i} \PY{o+ow}{in} \PY{n+nb}{range}\PY{p}{(}\PY{n+nb}{len}\PY{p}{(}\PY{n}{names}\PY{p}{)}\PY{p}{)}\PY{p}{:}
             \PY{n+nb}{print} \PY{p}{(}\PY{n+nb}{str}\PY{p}{(}\PY{n}{names}\PY{p}{[}\PY{n}{i}\PY{p}{]}\PY{p}{)} \PY{o}{+} \PY{l+s+s2}{\PYZdq{}}\PY{l+s+s2}{ is }\PY{l+s+s2}{\PYZdq{}} \PY{o}{+} \PY{n+nb}{str}\PY{p}{(}\PY{n}{ages}\PY{p}{[}\PY{n}{i}\PY{p}{]}\PY{p}{)} \PY{o}{+} \PY{l+s+s2}{\PYZdq{}}\PY{l+s+s2}{ years old.}\PY{l+s+s2}{\PYZdq{}}\PY{p}{)}
\end{Verbatim}


    \begin{Verbatim}[commandchars=\\\{\}]
Robert is 26 years old.
John is 31 years old.
Sarah is 29 years old.
Qian is 24 years old.
Ahmad is 30 years old.

    \end{Verbatim}

    \paragraph{Conditionals}\label{conditionals}

    Sometimes we want to check something before deciding what to do next.
For example,

    \begin{Verbatim}[commandchars=\\\{\}]
{\color{incolor}In [{\color{incolor}39}]:} \PY{k}{def} \PY{n+nf}{is\PYZus{}best\PYZus{}prof}\PY{p}{(}\PY{n}{name}\PY{p}{)}\PY{p}{:}
             \PY{k}{if} \PY{n}{name} \PY{o}{==} \PY{l+s+s2}{\PYZdq{}}\PY{l+s+s2}{Adam}\PY{l+s+s2}{\PYZdq{}}\PY{p}{:}
                 \PY{k}{return} \PY{l+s+s2}{\PYZdq{}}\PY{l+s+s2}{Yes!}\PY{l+s+s2}{\PYZdq{}}
             \PY{k}{else}\PY{p}{:}
                 \PY{k}{return} \PY{l+s+s2}{\PYZdq{}}\PY{l+s+s2}{No!}\PY{l+s+s2}{\PYZdq{}}
\end{Verbatim}


    \begin{Verbatim}[commandchars=\\\{\}]
{\color{incolor}In [{\color{incolor}42}]:} \PY{n+nb}{print} \PY{p}{(}\PY{n}{is\PYZus{}best\PYZus{}prof}\PY{p}{(}\PY{l+s+s2}{\PYZdq{}}\PY{l+s+s2}{Adam}\PY{l+s+s2}{\PYZdq{}}\PY{p}{)}\PY{p}{)}
\end{Verbatim}


    \begin{Verbatim}[commandchars=\\\{\}]
Yes!

    \end{Verbatim}

    \begin{Verbatim}[commandchars=\\\{\}]
{\color{incolor}In [{\color{incolor}44}]:} \PY{n+nb}{print} \PY{p}{(}\PY{n}{is\PYZus{}best\PYZus{}prof}\PY{p}{(}\PY{l+s+s2}{\PYZdq{}}\PY{l+s+s2}{John}\PY{l+s+s2}{\PYZdq{}}\PY{p}{)}\PY{p}{)}
\end{Verbatim}


    \begin{Verbatim}[commandchars=\\\{\}]
No!

    \end{Verbatim}

    \paragraph{Packages}\label{packages}

    Python has a ton of packages that make doing complicated stuff very
easy. We won't discuss how to install packages, or give a detailed list
of what packages exist, but we will give a brief description about how
they are used. An easy way to think of why package are useful is by
thinking: "Python packages give us access to MANY functions!"

In this class we will use four packages very frequently: pandas,
sklearn, matplotlib, and numpy:

\begin{itemize}
\tightlist
\item
  pandas is a data manipulation package. It let's you store data in data
  frames. More on this next class.
\item
  sklearn is a machine learning and data science package. It let's you
  do fairly complicated machine learning tasks, such as running
  regressions and building classification models with only a few lines
  of code!
\item
  matplotlib let's you make nice looking plots.
\item
  numpy (pronounced num-pie) is used for doing "math stuff" such as
  complex math operations (e.g., square roots, exponents, logs) and give
  you complex matrix operation abilities. If it's confusing as to why
  this is useful, don't worry. As we use them throughout the semester,
  their usefulness will become apparent.
\end{itemize}

To make the contents of a package useful, you need to import it:

    \begin{Verbatim}[commandchars=\\\{\}]
{\color{incolor}In [{\color{incolor}48}]:} \PY{k+kn}{import} \PY{n+nn}{pandas}
         \PY{k+kn}{import} \PY{n+nn}{sklearn}
         \PY{k+kn}{import} \PY{n+nn}{matplotlib}
         \PY{k+kn}{import} \PY{n+nn}{numpy}
\end{Verbatim}


    We can now use some package specific things. For example, numpy has a
function called sqrt() which will give us the square root of a numpy.
Since it is part of numpy, we need to tell Python that's where it is by
using a dot.

    \begin{Verbatim}[commandchars=\\\{\}]
{\color{incolor}In [{\color{incolor}49}]:} \PY{n}{numpy}\PY{o}{.}\PY{n}{sqrt}\PY{p}{(}\PY{l+m+mi}{25}\PY{p}{)}
\end{Verbatim}


\begin{Verbatim}[commandchars=\\\{\}]
{\color{outcolor}Out[{\color{outcolor}49}]:} 5.0
\end{Verbatim}
            
    You may have noticed that earlier, when we added stuff to our list, we
used .append(). This is very similar! Here, we told Python that numpy
had a function called sqrt() that we would like to use. Earlier, we told
Python that our list (and all lists!) had a function called append()
that we would like to use.

That's all we say about packages for now. Soon, we will be using
packages in every class. With practice, you will understand why they are
so great!

    \subsubsection{IPython Notebooks}\label{ipython-notebooks}

    IPython notebooks make writing Python code easy and neat.

    \paragraph{Auto complete}\label{auto-complete}

    One of the most useful things about IPython notebook is its tab
completion.

Try this: click just after numpy. in the cell below and press Tab
several times, slowly to view candidate functions you might want to call

    \begin{Verbatim}[commandchars=\\\{\}]
{\color{incolor}In [{\color{incolor} }]:} \PY{n}{numpy}\PY{o}{.}
\end{Verbatim}


    \paragraph{Organisation}\label{organisation}

    We typically read IPython notebooks from top to bottom. This means that
if a cell relies on a variable or function that was created earlier in
the notebook, you must run the corresponding cell to make that
information available! For example, if I set the variable age equal to
26 in the next cell,

    \begin{Verbatim}[commandchars=\\\{\}]
{\color{incolor}In [{\color{incolor}53}]:} \PY{n}{age} \PY{o}{=} \PY{l+m+mi}{26}
\end{Verbatim}


    but don't run it, it will not be available in the next cell:

    \begin{Verbatim}[commandchars=\\\{\}]
{\color{incolor}In [{\color{incolor}55}]:} \PY{n+nb}{print} \PY{p}{(}\PY{l+s+s2}{\PYZdq{}}\PY{l+s+s2}{I am }\PY{l+s+s2}{\PYZdq{}} \PY{o}{+} \PY{n+nb}{str}\PY{p}{(}\PY{n}{age}\PY{p}{)} \PY{o}{+} \PY{l+s+s2}{\PYZdq{}}\PY{l+s+s2}{ years old!}\PY{l+s+s2}{\PYZdq{}}\PY{p}{)}
\end{Verbatim}


    \begin{Verbatim}[commandchars=\\\{\}]
I am 26 years old!

    \end{Verbatim}

    Now, this does not mean you have to run everything from top to bottom.
You could define age later in the notebook, and then scroll up and run
any other cells that require it. However, this is bad practice. How is
someone supposed to know to scroll down first!? The IPython notebook
will always remember the cells in the order you run them. This means you
can overwrite variables!

    \begin{Verbatim}[commandchars=\\\{\}]
{\color{incolor}In [{\color{incolor}57}]:} \PY{n}{gender} \PY{o}{=} \PY{l+s+s2}{\PYZdq{}}\PY{l+s+s2}{male}\PY{l+s+s2}{\PYZdq{}}
\end{Verbatim}


    \begin{Verbatim}[commandchars=\\\{\}]
{\color{incolor}In [{\color{incolor}58}]:} \PY{n+nb}{print} \PY{p}{(}\PY{n}{gender}\PY{p}{)}
\end{Verbatim}


    \begin{Verbatim}[commandchars=\\\{\}]
male

    \end{Verbatim}

    \begin{Verbatim}[commandchars=\\\{\}]
{\color{incolor}In [{\color{incolor}61}]:} \PY{n}{gender} \PY{o}{=} \PY{l+s+s2}{\PYZdq{}}\PY{l+s+s2}{female}\PY{l+s+s2}{\PYZdq{}}
\end{Verbatim}


    \begin{Verbatim}[commandchars=\\\{\}]
{\color{incolor}In [{\color{incolor}62}]:} \PY{n+nb}{print} \PY{p}{(}\PY{n}{gender}\PY{p}{)}
\end{Verbatim}


    \begin{Verbatim}[commandchars=\\\{\}]
female

    \end{Verbatim}

    This can get confusing. Notice that the number in the "In {[}\#{]}:"
statement will always increase by one for every cell you run. This will
make keeping track of everything a little easier. But it is recommended
to always recommend organizing from top to bottom.

    \paragraph{Saving}\label{saving}

    There is an autosave feature and you can also use the save button in the
menu or toolbar.

    \subsubsection{Further Information}\label{further-information}

    The resources here are endless; but people learn things in different
ways. Some people prefer books, others like being taught through
lessons, and others just learn by doing. I, unfortunately, learn by
doing which means that I'm not an expert on which resources are the
best.

\begin{itemize}
\tightlist
\item
  icourse163.org has python courses.
\item
  Codecademy's Python Course will give you a great foundation for
  Python.
\item
  The online book, Diving into Python, has proven to be useful for many
  people.
\item
  ...
\item
  Personally, I try to think of a cool project I want to work on and I
  just try to do it. For example, trying to build a simple interactive
  text game can be fun. Since you are just jumping in and doing Python,
  this will lead to a ton of furious searching online. This is normal.
  This is actually how professional programmers work every day...
  Stackoverflow.com for instance has an answer to any programming
  question you can imagine!
\end{itemize}

    \subsubsection{Coding Questions}\label{coding-questions}

    To master your new found knowledge of Python, you should try these
hands-on examples.

    1. Create a list of 5 fruits (make sure to include an apple).

    \begin{Verbatim}[commandchars=\\\{\}]
{\color{incolor}In [{\color{incolor}17}]:} \PY{n}{my\PYZus{}list} \PY{o}{=} \PY{p}{[}\PY{l+s+s2}{\PYZdq{}}\PY{l+s+s2}{apple}\PY{l+s+s2}{\PYZdq{}}\PY{p}{,} \PY{l+s+s2}{\PYZdq{}}\PY{l+s+s2}{pear}\PY{l+s+s2}{\PYZdq{}}\PY{p}{,} \PY{l+s+s2}{\PYZdq{}}\PY{l+s+s2}{orange}\PY{l+s+s2}{\PYZdq{}}\PY{p}{,} \PY{l+s+s2}{\PYZdq{}}\PY{l+s+s2}{pinapple}\PY{l+s+s2}{\PYZdq{}}\PY{p}{,}\PY{l+s+s2}{\PYZdq{}}\PY{l+s+s2}{mandarin}\PY{l+s+s2}{\PYZdq{}}\PY{p}{]}
         \PY{n}{my\PYZus{}list}
\end{Verbatim}


\begin{Verbatim}[commandchars=\\\{\}]
{\color{outcolor}Out[{\color{outcolor}17}]:} ['apple', 'pear', 'orange', 'pinapple', 'mandarin']
\end{Verbatim}
            
    2. Go through each fruit and check if it is an apple. If it is, print
out "I found it!". If it's not an apple, do nothing.

    \begin{Verbatim}[commandchars=\\\{\}]
{\color{incolor}In [{\color{incolor}18}]:} \PY{k}{for}  \PY{n}{fruit} \PY{o+ow}{in} \PY{n}{my\PYZus{}list} \PY{p}{:}
             \PY{k}{if} \PY{p}{(}\PY{n}{fruit} \PY{o}{==} \PY{l+s+s2}{\PYZdq{}}\PY{l+s+s2}{apple}\PY{l+s+s2}{\PYZdq{}}\PY{p}{)}\PY{p}{:}
                 \PY{n+nb}{print} \PY{p}{(}\PY{l+s+s2}{\PYZdq{}}\PY{l+s+s2}{found it!}\PY{l+s+s2}{\PYZdq{}}\PY{p}{)}
\end{Verbatim}


    \begin{Verbatim}[commandchars=\\\{\}]
found it!

    \end{Verbatim}

    3. Add two new fruits to your list.

    \begin{Verbatim}[commandchars=\\\{\}]
{\color{incolor}In [{\color{incolor}19}]:} \PY{n}{my\PYZus{}list}\PY{o}{.}\PY{n}{append}\PY{p}{(}\PY{l+s+s2}{\PYZdq{}}\PY{l+s+s2}{bannana}\PY{l+s+s2}{\PYZdq{}}\PY{p}{)}
         \PY{n}{my\PYZus{}list}\PY{o}{.}\PY{n}{append}\PY{p}{(}\PY{l+s+s2}{\PYZdq{}}\PY{l+s+s2}{kiwi}\PY{l+s+s2}{\PYZdq{}}\PY{p}{)}
         \PY{n}{my\PYZus{}list}
\end{Verbatim}


\begin{Verbatim}[commandchars=\\\{\}]
{\color{outcolor}Out[{\color{outcolor}19}]:} ['apple', 'pear', 'orange', 'pinapple', 'mandarin', 'bannana', 'kiwi']
\end{Verbatim}
            
    4. Create a new empty list. Go through your list of fruits, and for each
one, add an entry to the new list that tells us how many letters each
fruit name is.

    \begin{Verbatim}[commandchars=\\\{\}]
{\color{incolor}In [{\color{incolor}34}]:} \PY{k}{for}  \PY{n}{fruit} \PY{o+ow}{in} \PY{n}{my\PYZus{}list} \PY{p}{:}
             \PY{n+nb}{print} \PY{p}{(}\PY{l+s+s2}{\PYZdq{}}\PY{l+s+si}{\PYZob{}:s\PYZcb{}}\PY{l+s+s2}{ has }\PY{l+s+si}{\PYZob{}:d\PYZcb{}}\PY{l+s+s2}{ letters}\PY{l+s+s2}{\PYZdq{}}\PY{o}{.}\PY{n}{format}\PY{p}{(}\PY{n}{fruit}\PY{p}{,}\PY{n+nb}{len}\PY{p}{(}\PY{n}{fruit}\PY{p}{)}\PY{p}{)}\PY{p}{)}
\end{Verbatim}


    \begin{Verbatim}[commandchars=\\\{\}]
apple has 5 letters
pear has 4 letters
orange has 6 letters
pinapple has 8 letters
mandarin has 8 letters
bannana has 7 letters
kiwi has 4 letters

    \end{Verbatim}

    5. Make a function called half\_squared that takes a list and returns a
new list where each element of the original is squared and then divided
in half.

    \begin{Verbatim}[commandchars=\\\{\}]
{\color{incolor}In [{\color{incolor}35}]:} \PY{k}{def} \PY{n+nf}{half\PYZus{}squared}\PY{p}{(}\PY{n}{input\PYZus{}list}\PY{p}{)}\PY{p}{:}
             \PY{n}{output\PYZus{}list} \PY{o}{=} \PY{p}{[}\PY{p}{]} \PY{c+c1}{\PYZsh{} What should we do?}
             \PY{k}{for} \PY{n}{element} \PY{o+ow}{in} \PY{n}{input\PYZus{}list}\PY{p}{:}
                 \PY{n}{output\PYZus{}list}\PY{o}{.}\PY{n}{append}\PY{p}{(}\PY{p}{(}\PY{n}{element}\PY{o}{*}\PY{n}{element}\PY{p}{)}\PY{o}{/}\PY{l+m+mi}{2}\PY{p}{)}
             \PY{k}{return} \PY{n}{output\PYZus{}list}
\end{Verbatim}


    \begin{Verbatim}[commandchars=\\\{\}]
{\color{incolor}In [{\color{incolor}36}]:} \PY{c+c1}{\PYZsh{}\PYZsh{} test the function}
         \PY{n}{half\PYZus{}squared}\PY{p}{(}\PY{p}{[}\PY{l+m+mi}{3}\PY{p}{,}\PY{l+m+mi}{3}\PY{p}{]}\PY{p}{)} \PY{o}{==} \PY{p}{[}\PY{l+m+mf}{4.5}\PY{p}{,}\PY{l+m+mf}{4.5}\PY{p}{]}
\end{Verbatim}


\begin{Verbatim}[commandchars=\\\{\}]
{\color{outcolor}Out[{\color{outcolor}36}]:} True
\end{Verbatim}
            

    % Add a bibliography block to the postdoc
    
    
    
    \end{document}
